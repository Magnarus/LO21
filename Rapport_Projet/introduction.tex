\chapter{Introduction}
Le projet développé consiste en un mélange entre un agenda et une application de gestion de projets. Cette application est donc composée 
de deux parties: une partie permettant de visualiser son emploi du temps, et permettant l'ajout de programmation relative à des activités classiques (Rendez-Vous, Anniversaire, ect...) mais aussi de programmer les taches d'un projet, et une seconde partie représentant les projets 
saisis et permettant leur gestion. Il est nécessaire également, dans une optique de stockage des données et d'interopérabilité, de pouvoir exporter et importer les programmations existantes (ce qui nécessite l'exportation d'autres éléments annexes, afin de conserver une cohérence des données).
Au delà des fonctionnalités, le projet s'axe grandement sur le choix de l'architecture, afin de faciliter la maintenabilité de l'application et son évolution. Nous allons donc expliquer nos choix concernant l'architecture, puis en quoi celle-ci permet aisément l'évolution du projet.